\documentclass{article}
%\documentclass[10pt,fromemail=on,fromphone=on]{scrlttr2}
\usepackage[a4paper]{geometry}
\usepackage[utf8]{inputenc}
\usepackage[T1]{fontenc}
\usepackage{lmodern}
\usepackage[ngerman]{babel}
\usepackage{parskip}

\pagenumbering{gobble}

\begin{document}
\date{}
	\title{Dokumentation der täglichen Arbeitszeit nach §17
    Mindestlohngesetz für geringfügig Beschäftigte}
\maketitle
Das Mindestlohngesetz verlangt eine Dokumentation von Beginn, Ende und
Dauer der täglichen Arbeitszeit von geringfügig Beschäftigten. Die Zeiten sind bis
spätestens zum Ablauf des siebten auf den Tag der Arbeitsleistung folgenden
Kalendertages aufzuzeichnen. Sie sind mindestens zwei Jahre beginnend ab dem für
die Aufzeichnung maßgeblichen Zeitpunkt aufzubewahren.

  \vspace{0.5cm}
Geringfügig Beschäftigte/r: 
  \vspace{0.5cm}

  \begin{table}[h]
    \centering
    \begin{tabular}{l| c c c}
      \input{2018_06_tableData}
    \end{tabular}
  \end{table}

  \vspace{0.5cm}

  \noindent Hiermit bestätige ich, dass ich zu den oben aufgeführten oder äquivalenten Zeiten gearbeitet habe.

  \vspace{1.5cm}

  \begin{tabular}{lp{2em}l}
    \hspace{5cm}   && \hspace{7cm} \\\cline{1-1}\cline{3-3}
    Ort, Datum     && Unterschrift Beschäftigte/r
  \end{tabular}
 
  \vspace{1.5cm}

  \begin{tabular}{lp{2em}l}
    \hspace{5cm}   && \hspace{7cm} \\\cline{1-1}\cline{3-3}
    Ort, Datum     && Unterschrift Dienststelle
  \end{tabular}
  
\end{document}



%%% Local Variables:
%%% mode: latex
%%% TeX-master: t
%%% End:
